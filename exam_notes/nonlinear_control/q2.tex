\chapter{Question 2}
\assignment{
Consider the system:
\begin{equation}
\begin{split}
        \dot{x}_1 = -x_1 +x_2 - x_3 \\
        \dot{x}_2 = - x_1 x_3 - x_2 + u \\
        \dot{x}_3 = - x_1 + u \\
        y = x_3 
\end{split}
\end{equation}

a) Is the system Input-Output linearizable \\
b) If yes, transform it into the normal form and specify the region over which the transformation is valid \\
c) is the system minimum phase \\
}

\assignment{
        a) Is the system Input-Output linearisable?
}
We need to verify that we are able to linearise the \textbf{output} $y$ through the \textbf{input} $u$. This is done by investigating the \textit{relative degree} $\rho$, where $1 \leq \rho \leq n$, in a region $D_0 \subset D_0$

\begin{equation}
        \dot{y} = -x_1 + u \quad \rightarrow \quad \rho = 1
\end{equation}

which makes the system linearisable.

\assignment{
b) If yes, transform it into the normal form and specify the region over which the transformation is valid \\
}
We want to transform the Non-linear system to a linear system on the normal form:
\begin{equation}
        \begin{split}
                \dot{\eta} &= f_0(\eta,\xi) \\
                \dot{\xi} &= A_c \xi + B_c \gamma(x) \left[u - \alpha(x)\right] \\
                y &= C_c \xi
        \end{split}
\end{equation}
where $z = T(x)$ is a change of variables. The transform $T(x)$ should be a continuously differential map with a continuously differential inverse also known as a \textit{diffeomorphism}.

\begin{equation}
        z = T(x) = 
        \begin{pmatrix}
                \phi_1(x) \\
                \vdots \\
                \phi_{n - \rho}(x) \\
                --- \\
                y(x) \\
                \vdots \\
                y^{\rho - 1}(x)
        \end{pmatrix}
        \doteq
        \begin{pmatrix}
                \eta \\
                --- \\
                \xi
        \end{pmatrix}
\end{equation}
Where $\phi_1$ to $\phi_{n - \rho}$ is chosen such that $T(x)$ is a diffeomorphism on a domain $D_0 \subset D$ and 
\begin{equation}
        \frac{\partial \phi_{i}}{\partial x}g(x) = 0
\end{equation}
In the assignment the system is given:
\begin{equation}
        \dot{x} = f(x) + g(x)u
\end{equation}
where 
\begin{equation}
        f(x) = 
        \begin{bmatrix}
                -x_1 + x_2 - x_3 \\
                -x_1 x_3 - x_2 \\
                -x_1
        \end{bmatrix}
        ,\quad 
        g(x) = 
        \begin{bmatrix}
                0 \\
                1 \\
                1
        \end{bmatrix}
\end{equation}
where 
\begin{equation}
        \frac{\partial \phi_{i}}{\partial x}g(x) = \frac{\partial \phi_{i}}{\partial x} 
        \begin{bmatrix}
                0 \\
                1 \\
                1
        \end{bmatrix}
        =
        \begin{Bmatrix}
                \frac{\partial \phi_{1}}{\partial x_2} + \frac{\partial \phi_{1}}{\partial x_{3}} = 0 \\
                \frac{\partial \phi_{2}}{\partial x_2} + \frac{\partial \phi_{2}}{\partial x_{3}} = 0
        \end{Bmatrix}
\end{equation}
we choose the $\phi_i$ such that it complies with the above.
\begin{equation}
        \begin{split}
                \phi_1 &= x_1 \\
                \phi_2 &= x_2 - x_3 = x_3 - x_2 \\
                y &= x_3
        \end{split}
\end{equation}
\begin{equation}
        T(x) = 
        \begin{bmatrix}
                x_1 \\
                x_2 - x_3 \\
                x_3
        \end{bmatrix}
        =
        \begin{bmatrix}
                \eta_1 \\
                \eta_2 \\
                \xi
        \end{bmatrix}
\end{equation}

We can now transform the system to the normal form.
\begin{equation}
        \begin{split}
        \dot{\eta}_{1} &= \dot{x}_{1} = -x_1 + x_2 - x_3 = -\eta_1 + \eta_2 \\
        \dot{\eta}_2 &= \dot{x}_2 - \dot{x}_3 = -x_1 x_3 - x_2 + u - (-x_1 + u) = -\xi \eta_1 - \xi - \eta_2 + \eta_1 \\
        \dot{\xi} &= \dot{x}_3 = -\eta_1 + u
        \end{split}
\end{equation}
where $\xi$ is the external part of the system and $\eta$ is the internal part, which in relation to the normal form becomes
\begin{equation}
        \dot{\xi} = 0\xi + 1\left[u - \eta_1\right]
\end{equation}
Where the external part is linearised by
\begin{equation}
        u = \alpha(x) + \gamma^{-1} (x) v
\end{equation}

\assignment{
c) is the system minimum phase    
}
\textit{The system is said to be minimum phase if $\dot{\eta} = f_0(\eta,0)$ has an asymtotically stable equilibrium point in the domain of interest.}

We could also say that all the zeros of the system should be in the open left half plane.

\begin{equation}
        \dot{\eta} = f_0(\eta,0) = 
        \begin{bmatrix}
                -1 & 1 \\
                1 & -1
        \end{bmatrix}
        \begin{bmatrix}
                \eta_1 \\
                \eta_2
        \end{bmatrix}
\end{equation}