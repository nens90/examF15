\chapter{Question 2}
\assignment{
Consider the system:
\begin{equation}
\begin{split}
        \dot{x}_1 = -x_1 +x_2 - x_3 \\
        \dot{x}_2 = - x_1 x_2 - x_2 + u \\
        \dot{x}_3 = - x_1 + u \\
        y = x_3 
\end{split}
\end{equation}

a) Is the system Input-Output linearizable \\
b) If yes, transform it into the normal form and specify the region over which the transformation is valid \\
c) is the system minimum phase \\
}

\assignment{
        a) Is the system Input-Output linearisable?
}
We need to verify that we are able to linearise the \textbf{output} $y$ through the \textbf{input} $u$. This is done by investigating the \textit{relative degree} $\rho$, where $1 \leq \rho \leq n$, in a region $D_0 \subset D_0$

\begin{equation}
        \dot{y} = -x_1 + u \quad \rightarrow \quad \rho = 1
\end{equation}

which makes the system linearisable.

\assignment{
b) If yes, transform it into the normal form and specify the region over which the transformation is valid \\
}
We want to transform the Non-linear system to a linear system on the normal form:
\begin{equation}
        \dot{z} = Az + B\gamma(x) \left[u-\alpha(x)\right]
\end{equation}
where $z = T(x)$ is a change of variables. The transform $T(x)$ should be a continuously differential map with a continuously differential inverse also known as a \textit{diffeomorphism}.

\begin{equation}
        z = T(x) = 
        \begin{pmatrix}
                \phi_1(x) \\
                \vdots \\
                \phi_{n - \rho}(x) \\
                --- \\
                y(x) \\
                \vdots \\
                y^{\rho - 1}(x)
        \end{pmatrix}
        \doteq
        \begin{pmatrix}
                \eta \\
                --- \\
                \xi
        \end{pmatrix}
\end{equation}
Where $\phi_1$ to $\phi_{n - \rho}$ is chosen such that $T(x)$ is a diffeomorphism on a domain $D_0 \subset D$ and 
\begin{equation}
        \frac{\partial \phi_{i}}{\partial x}g(x) = 0
\end{equation}
\assignment{
c) is the system minimum phase    
}