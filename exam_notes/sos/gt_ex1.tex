\chapter{Game Theory Exercise 1}
% \assignment{ 
\begin{itemize}
\item Z. Han, D. Niyato, W. Saad, T. Basar, A. Hjørungnes. Game Theory in Wireless and Communication Networks:
  Theory, Models, and Applications. Cambridge University Press 2012. Chapters 3.1, 3.2
\item A. MacKenzie, L. DaSilva. Game Theoryfor Wireless Engineers. Morgan \& Claypool Publishers. 2006. Chapter
  3.
\end{itemize} 
% }

\assignment{
  \textbf{Problem 1 - Prisoners Dilemma} \\
  Two suspects are arrested for a crime and placed in two isolated rooms. Each ones of the suspects has 
  to decide whether or not to confess and implicate the other. The rules are the following. If none of the 
  suspects confesses, then each will serve 2 years in jail. If both of them confess and implicate each other, 
  they will both go to prison for 4 years. However, if one prisoner confesses and implicates the other while 
  the other one does not confess, the one who has cooperated with the police will be set free, while the 
  other will spend 5 years in prison.\\
  % Define who players in this game are and what the possible strategies are. Formulate the game in 
  % strategic form (give a matrix representation of the game). Is this game a zero‐sum game or not‐zero‐
  % sum game? Are there any dominating strategies? Find Nash equilibrium.
  \begin{enumerate}
  \item Define who are the players
  \item What are the possible strategies
  \item Formulate in matrix representation.
  \item Is the game zero-sum game?
  \item Are there any dominating strategies?
  \item Find Nash the equilibrium.
  \end{enumerate}
}

\assignment{1 .Define who are the players:} 
There are two players in the game - The two prisoners. The game is a non-coorporative game.

\assignment{2. What are the possible strategies:} 
The strategy space $S = (0,1)$
  where 0 is to deny and 1 is to confess. This gives the combinations:
  \begin{itemize}\tightlist
  \item (0,0): Both prisoners deny.
  \item (0,1): Prisoner 1 denys and prisoner 2 confesses.
  \item (1,0): Prisoner 1 confesses and prisoners 2 denys.
  \item (1,1): Both prisoners confesses.
  \end{itemize}

\assignment{3. Formulate in matrix representation:} 
The matrix representation
  \begin{center}
    \begin{tabular}{ r|c|c| }
      \multicolumn{1}{r}{}
      & \multicolumn{1}{c}{$s_2 = 0$}
      & \multicolumn{1}{c}{$s_2 = 1$}                                   \\
      \cline{2-3}
      $s_1 = 0$ & $(2,2)$ & $(5,0)$                                               \\
      \cline{2-3}
      $s_1 = 1$ & $(0,5)$ & $(4,4)$                                               \\
      \cline{2-3}
    \end{tabular}
  \end{center}

\assignment{4. Is the game zero-sum game:} No - the game is a nonzero-sum game. A zero-sum game is a game where - if one
player gains - the other player must lose the same amount of ``cost''. An example of a sero-sum game is the
rock-paper-scissor game where if one player wins the other loses and there are no individual cost; just win or lose.

\assignment{5. Are there any dominating strategies:} The dominating strategies is to confess $s_i = 1$. Because that each
player gains from confessing this is the dominant strategy-

\assignment{6. Find Nash the equilibrium:} The Nash equilibrium is $(4,4)$, this is because of the fact that none of the
players can gain from changing strategy at this point taking the other players choice in consideration. If whichever of
the prisoners changes strategy from this point it will cause a less favorable outcome for the prisoner in question.

  There is a global minimum at (2,2), but this is an unstable minimum where both of the prisoners can gain, from
changing strategy.
