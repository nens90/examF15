\documentclass[a4paper,11pt,openright]{memoir}

% Pakker til brug i dokumentet:
\usepackage[english]{babel}
\usepackage[utf8]{inputenc}
\usepackage[T1]{fontenc}
\usepackage{graphicx} % Indsættelse af billeder mm.
\graphicspath{{../figures/}}
\usepackage{float}
\usepackage{amsmath,amssymb,mathtools,bm} % Diverse matematik
\usepackage{color,xcolor}
\usepackage{pdfpages} % Giver mulighed for at sætte hele pdf sider ind
\usepackage{listings} % Præsentering af kode
\usepackage{booktabs}
\usepackage{multirow}
\usepackage{longtable}
\usepackage{pbox}
\usepackage{pgfplots}
\usepackage{tikz}
\usetikzlibrary{shapes,arrows} % Libaries til brug i tikz
\usepackage[pdftitle={CubeSat Attitude Control with Supervisory Control System},
            pdfauthor={Group 930,
               Aalborg University,
               Aalborg,
               Denmark},
            pdfduplex=DuplexFlipLongEdge,
            colorlinks=false,
            hidelinks=true,
            pdftex,
            pdfmenubar=true,
            pdftoolbar=true,
            pdfstartview={FitH}
            ]{hyperref}
\usepackage{memhfixc} %include after hyperref when using memoir documentclass
\usepackage{datatool}
\usepackage[acronym,toc,nonumberlist]{glossaries}
\newglossary[slg]{symbols}{sym}{sbl}{List of Symbols}
\usepackage[hang]{caption}
\captionsetup{font=footnotesize,labelfont={bf},textfont=normalfont}
\setlength{\captionmargin}{20pt}
\usepackage{epstopdf}
\epstopdfsetup{update}
\usepackage{lipsum}
\usepackage[textsize=footnotesize]{todonotes}
\usepackage{lscape}
\usepackage{ushort}

% Setup af dokumentet:
\makeatletter
\renewcommand{\@chapapp}{Worksheet}
\makeatother

\chapterstyle{madsen} % Kapitel layoutet - ell, madsen
\pagestyle{ruled} % Header og Footer - "plain" for intet, ud over sidetal i bunden
\setsecnumdepth{subsection} % Hvor dybt de forskellige sections skal nummereres
\settocdepth{section} % "Table of Contents" dybden
\pretolerance=2500 % Sætter tolerancen for orddeling; jo højere, jo mindre orddeling
\hyphenation{hvem hvad hvor} % Fortæller latex hvordan den må dele et ord; kun relevant hvis et ord bliver delt forkert.
\setlength{\parindent}{1.5mm} % Størrelse af indryk ved nyt afsnit
\setlength{\parskip}{1.8mm} % Afstand mellem afsnit ved brug af "double Enter"
\setcounter{MaxMatrixCols}{20} % Max antal kolonner i en tabel
\addto\captionsenglish{
  \renewcommand{\contentsname}
    {Table of Contents}
}
\newlength\figureheight 
\newlength\figurewidth 

% Penalties for blandt andet horeunger og enker.
\clubpenalty=9996
\widowpenalty=9999
\brokenpenalty=4991
\predisplaypenalty=10000
\postdisplaypenalty=1549
\displaywidowpenalty=1602
\raggedbottom

% Margin
\setlrmarginsandblock{3.5cm}{2.5cm}{*} % \setlrmarginsandblock{Indbinding}{Kant}{Ratio}
\setulmarginsandblock{2.5cm}{3.0cm}{*} % \setulmarginsandblock{Top}{Bund}{Ratio}

% Farver til listset
\definecolor{dkgreen}{rgb}{0,0.6,0} % Definerer farven grøn til brug i præsentation af kildekode
\definecolor{gray}{rgb}{0.5,0.5,0.5} % Definerer farven grå til brug i præsentation af kildekode
\definecolor{mauve}{rgb}{0.58,0,0.82} % Definerer farven pink til brug i præsentation af kildekode
\definecolor{darkblue}{rgb}{0.0,0.0,0.6}

% Listset til præsentation af kildekode
\lstset{frame=tb,
  language=C,
  aboveskip=3mm,
  belowskip=3mm,
  showstringspaces=false,
  columns=flexible,
  basicstyle={\footnotesize\ttfamily}, % basicstyle={\small\ttfamily},
  stepnumber=2,
  firstnumber=1,
  escapeinside={@}{@},
  numberfirstline=false,
  numbers=left,
  numbersep=8pt,
  numberstyle=\tiny\color{gray},
  keywordstyle=\color{blue},
  commentstyle=\color{dkgreen},
  stringstyle=\color{mauve},
  breaklines=true,
  breakatwhitespace=true
  tabsize=3,
  captionpos=b,
  keepspaces=true
}

\lstdefinelanguage{XML} {
  morestring=[b]",
  morestring=[s]{>}{<},
  morecomment=[s]{<?}{?>},
  identifierstyle=\color{darkblue},
  stringstyle=\color{black},
  morekeywords={xmlns,version,type}
}

\lstdefinelanguage{verb} {
  columns=fixed,
  fontadjust=true,
  basewidth=0.5em
}

\checkandfixthelayout % Laver forskellige beregninger og sætter de almindelige længder op
